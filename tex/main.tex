%% Vorlage Bachelorarbeit

%% Versionshistorie:

%% v1.0: Erstellung durch Johannes Woske, IT2010
%% v2.0: Überarbeitung und Ergänzung durch Anne Traulsen, IT2015
%% v3.0: work in progress 

\documentclass[
	12pt, %Schriftgröße
	a4paper,
	liststotoc, %Inhaltsverzeichniseinträge für Listen (z.B. Abbildungen)
	bibtotoc, %Inhaltsverzeichniseinträge f+r Quellen
	pointlessnumbers, %Entfernt Punkt hinter Gliederungsnummern
	ngerman, %Sprachpaket
	headsepline, %Headertrennlinie
	%footsepline, %Footertrennlinie
	oneside %einseitiges Druckformat %%% Unterdrücken der leeren Seite nach Titelblatt
	]{scrbook} %Dokumentenklasse (Koma-Script)
\usepackage[T1]{fontenc}
\usepackage{float}
\usepackage[utf8]{inputenc}
\usepackage[ngerman]{babel}
\usepackage{lmodern}
\usepackage{markdown}
\usepackage{fancyhdr}
\usepackage{chronology}
\usepackage{url}
\usepackage{graphicx} %Bilder einfügen
\usepackage{float}
\usepackage{microtype}
\usepackage[a4paper, margin=1in]{geometry}
\usepackage[right]{eurosym} %Euro-Zeichen
\usepackage{amssymb}
\usepackage{babel}
\usepackage{fontenc}
\usepackage{graphicx}
\usepackage{cite} %Quellenangaben
\usepackage{setspace} % Zeilenabstand
\usepackage[ 
   colorlinks,        % Links ohne Umrandungen in zu wählender Farbe 
   linkcolor=black,   % Farbe interner Verweise 
   filecolor=black,   % Farbe externer Verweise 
   citecolor=black,   % Farbe von Zitaten 
   urlcolor=blue	  % Farbe von Links
   ]{hyperref} %Verlinkungen
\usepackage[figure]{hypcap}
\usepackage[toc,page]{appendix}
\usepackage[ngerman]{translator}
\usepackage{listings,xcolor} %Codeanzeige

%Hinzufügen von C#
\lstdefinestyle{sharpc}{language=[Sharp]C, frame=lr, rulecolor=\color{blue!80!black}}
\usepackage[normalem]{ulem}
\useunder{\uline}{\ul}{}
\usepackage{pdfpages}
\usepackage{color}
\definecolor{lst-gray}{rgb}{0.98,0.98,0.98}
\definecolor{lst-blue}{RGB}{40,0.0,255}
\definecolor{lst-green}{RGB}{65,128,95}
\definecolor{lst-red}{RGB}{200,0,85}

\lstset{literate=
  {á}{{\'a}}1 {é}{{\'e}}1 {í}{{\'i}}1 {ó}{{\'o}}1 {ú}{{\'u}}1
  {Á}{{\'A}}1 {É}{{\'E}}1 {Í}{{\'I}}1 {Ó}{{\'O}}1 {Ú}{{\'U}}1
  {à}{{\`a}}1 {è}{{\`e}}1 {ì}{{\`i}}1 {ò}{{\`o}}1 {ù}{{\`u}}1
  {À}{{\`A}}1 {È}{{\'E}}1 {Ì}{{\`I}}1 {Ò}{{\`O}}1 {Ù}{{\`U}}1
  {ä}{{\"a}}1 {ë}{{\"e}}1 {ï}{{\"i}}1 {ö}{{\"o}}1 {ü}{{\"u}}1
  {Ä}{{\"A}}1 {Ë}{{\"E}}1 {Ï}{{\"I}}1 {Ö}{{\"O}}1 {Ü}{{\"U}}1
  {â}{{\^a}}1 {ê}{{\^e}}1 {î}{{\^i}}1 {ô}{{\^o}}1 {û}{{\^u}}1
  {Â}{{\^A}}1 {Ê}{{\^E}}1 {Î}{{\^I}}1 {Ô}{{\^O}}1 {Û}{{\^U}}1
  {Ã}{{\~A}}1 {ã}{{\~a}}1 {Õ}{{\~O}}1 {õ}{{\~o}}1
  {œ}{{\oe}}1 {Œ}{{\OE}}1 {æ}{{\ae}}1 {Æ}{{\AE}}1 {ß}{{\ss}}1
  {ű}{{\H{u}}}1 {Ű}{{\H{U}}}1 {ő}{{\H{o}}}1 {Ő}{{\H{O}}}1
  {ç}{{\c c}}1 {Ç}{{\c C}}1 {ø}{{\o}}1 {å}{{\r a}}1 {Å}{{\r A}}1
  {€}{{\euro}}1 {£}{{\pounds}}1 {«}{{\guillemotleft}}1
  {»}{{\guillemotright}}1 {ñ}{{\~n}}1 {Ñ}{{\~N}}1 {¿}{{?`}}1
}

\usepackage{color}

\definecolor{javared}{rgb}{0.6,0,0} % for strings
\definecolor{javagreen}{rgb}{0.25,0.5,0.35} % comments
\definecolor{javapurple}{rgb}{0.5,0,0.35} % keywords
\definecolor{javadocblue}{rgb}{0.25,0.35,0.75} % javadoc

%Hinzufügen von Java 
\lstset{language=Java,
basicstyle=\ttfamily,
keywordstyle=\color{javapurple}\bfseries,
stringstyle=\color{javared},
commentstyle=\color{javagreen},
morecomment=[s][\color{javadocblue}]{/**}{*/},
numbers=left,
numberstyle=\tiny\color{black},
stepnumber=2,
numbersep=10pt,
tabsize=4,
showspaces=false,
showstringspaces=false}

\usepackage{chngcntr}
\usepackage{verbatim}
\usepackage{wrapfig}
\counterwithout{figure}{chapter}
\counterwithout{table}{chapter}
\usepackage[toc, acronym, nonumberlist, ]{glossaries}
 
\definecolor{mygreen}{rgb}{0,0.6,0}
\definecolor{mygray}{rgb}{0.5,0.5,0.5}
\definecolor{mymauve}{rgb}{0.58,0,0.82}

\lstset{ %
  backgroundcolor=\color{white},   % choose the background color
  basicstyle=\footnotesize,        % size of fonts used for the code
  breaklines=true,                 % automatic line breaking only at whitespace
  captionpos=b,                    % sets the caption-position to bottom
  commentstyle=\color{mygreen},    % comment style
  escapeinside={\%*}{*)},          % if you want to add LaTeX within your code
  keywordstyle=\color{blue},       % keyword style
  stringstyle=\color{mymauve},     % string literal style
}
%%%%%%%%%%%%%%%%%%%%%%%%%%%%%%%%%%%%%%%%%%%%%%%%%%%%%
%%%%%%%%%%% Sonderformatierung
%%%%%%%%%%%%%%%%%%%%%%%%%%%%%%%%%%%%%%%%%%%%%%%%%%%%%

% Seitenabstände definieren
\geometry{verbose,tmargin=3cm,bmargin=2cm,lmargin=3cm,rmargin=3cm} 

% Hurenkinder und Schusterjungen verhindern (Ja, das heißt wirklich so!!)
\clubpenalty = 10000 \widowpenalty = 10000 \displaywidowpenalty = 10000 

\newcommand{\footfigref}[1]{\footnote{Abb. \ref{#1} auf Seite \pageref{#1}}}

%% Bei Referenzen im Text wird jetzt bei allen Ebenen "Kapitel" vorgestellt, z.b. Kapitel 2, Kapitel 2.2, Kapitel 6.3.2
\addto\extrasngerman{%
    \def\sectionautorefname{Kapitel}%
    \def\subsectionautorefname{Kapitel}%
    \def\subsubsectionautorefname{Kapitel}%
    }

% Vertikaler Abstand zwischen Ende Textblock - Ende Fußzeile --> Abstand der Seitenzahl von Rand erhöhen 
\setlength{\footskip}{10mm}

% Abstand vor/nach Überschriften verändern

\RedeclareSectionCommand[%
    beforeskip=0.5\baselineskip,
    afterskip=0.5\baselineskip
]{chapter}

\RedeclareSectionCommand[%
    beforeskip=0.5\baselineskip,
    afterskip=0.5\baselineskip
]{section}

\RedeclareSectionCommand[%
    beforeskip=0.1\baselineskip,
    afterskip=0.1\baselineskip
]{subsection}

\RedeclareSectionCommand[%
    beforeskip=0.01\baselineskip,
    %%afterskip=0.2\baselineskip
]{paragraph}

\setlength{\abovecaptionskip}{4pt}  % 1pc=12pt 
\setlength{\belowcaptionskip}{0pt}
%\setlength{\textfloatsep}{4pt}
\setlength{\intextsep}{1pc}

%% Verkleinerung der Textgröße unter Abbildungen
\addtokomafont{caption}{\small}

% falsche Default-Silbentrennung überschreiben
\include{hyphenation}

% Den Punkt am Ende der Glossareinträge deaktivieren
\renewcommand*{\glspostdescription}{}

%Glossar-Befehle anschalten

% sorgt dafür, dass bei Leerzeile die Einrückung verhindert und stattdessen eine Leerzeile eingefügt wird % erspart bigskips und erhöht die Lesbarkeit im LaTeX-Text 
\KOMAoptions{parskip=full*}

%Kopf und Fußzeile bearbeiten
\usepackage{fancyhdr}
 
\pagestyle{fancy}
\fancypagestyle{plain}{%
  \fancyhf{}
%Kopfzeile
\lhead{\includegraphics[scale=1]{assets/images/HWR_Logo_farbig.jpg}}
\chead{\thepage}
\rhead{\includegraphics[scale=0.05]{assets/images/DB_Logo.png}}
%Unterbindet den schwarzen unterstrich
\renewcommand{\headrulewidth}{0pt}
}
%so überschreibt es auch die anderen settings (keine ahnung wieso das so ist...)
\lhead{\includegraphics[scale=1]{assets/images/HWR_Logo_farbig.jpg}}
\chead{\thepage}
\rhead{\includegraphics[scale=0.05]{assets/images/DB_Logo.png}}
\cfoot{} %so wird verhindert, dass die Seitenanzahl auch unten angezeigt wird
%Unterbindet den schwarzen unterstrich
\renewcommand{\headrulewidth}{0pt}

\author{Maximilian Mustermann}

% ändert Titelschriftart in Serifen-Normalschriftart
\addtokomafont{disposition}{\rmfamily} 
%%%%%%%%%%%%%%%%%%%%%%%%%%%%%%%%%%%%%%%%%%%%%%%%%%%%%
%%%%%%%%%%% Textbausteine
%%%%%%%%%%%%%%%%%%%%%%%%%%%%%%%%%%%%%%%%%%%%%%%%%%%%%
%%%%%%%%%%%% Studentenname
\newcommand{\studentName}{Maximilian Mustermann}
%%%%%%%%%%%% Typ der Arbeit
\newcommand{\type}{Praxistransferbericht/Studienarbeit/BT}
%%%%%%%%%%%% Thema
\newcommand{\topic}{LaTex Vorlage für FB 2 HWR}
%%%%%%%%%%%% Untertitel
\newcommand{\subtopic}{Hier Untertitel eintragen}
%%%%%%%%%%%% Studienbereich
\newcommand{\fachbereich}{Duales Studium Wirtschaft · Technik}
%%%%%%%%%%%% Fachrichtung
\newcommand{\fachrichtung}{Wirtschaftsinformatik}
%%%%%%%%%%%% Betrieb
\newcommand{\company}{DB Systel GmbH}
%%%%%%%%%%%% Betreuer HWR
\newcommand{\betreuerHS}{Prof. Dr. Prüfer}
%%%%%%%%%%%% Betreuer Unternehmen
\newcommand{\betreuerUnt}{Dein Betreuer}
%%%%%%%%%%%% Jahrgang
\newcommand{\jahrgang}{2018}
%%%%%%%%%%%% Semester
\newcommand{\semester}{3}

% Glossar einbinden
%%%%%%%%%%%%%%%%%%%%%%%%%%%%%%%%%%%%%%%%%%%%
%%%%% Befehle für Abkürzungen 

\newacronym{DB}{DB}{Deutsche Bahn AG}

\newglossaryentry{BKU}{name=Bürokommunikation unternehmensweit, description={Bürokommunikation unternehmensweit ist die IT-Plattform zur Bürokommunikation der DB. \cite{BKU_def}}}
\makenoidxglossaries

%Dokument beginnt hier 
\begin{document}

%Die ersten Kapitel werden Römisch numeriert und werden (in diesem Beispiel)
%nicht mit ins Inhaltsverzeichnis aufgenommen
\pagenumbering{Roman}

% Titelseite
%%%%%%%%%%%%%%%%%%%%%%%%%%%%%%%%%%%%%%%%%%%%%%%%%%%%%>>>>>>>
%%%%%%%%%%% Titelblatt

%% Anordnung und Aussehen von Titel und Untertitel

\subject{\type}

\title{
\normalfont\endgraf\rule{\textwidth}{.4pt}
\begingroup
	\centering
	\linespread{1.5}
	\huge\topic\\
	\normalsize\subtopic
\endgroup
\endgraf\rule{\textwidth}{.4pt}
}
 
\date{\normalsize vorgelegt am tt.mm.jjjj\\
an der \\ 
Hochschule für Wirtschaft und Recht Berlin\vspace{5mm}}

\publishers{
	\begin{tabular}{l l}
	\textbf{\normalsize{Von: }} & \normalsize{\studentName} 
	\tabularnewline
	\textbf{\normalsize{Fachbereich:}} & \normalsize{\fachbereich}  \tabularnewline
	\textbf{\normalsize{Fachrichtung:}} & \normalsize{\fachrichtung} \tabularnewline
	\textbf{\normalsize{Studienjahrgang:}} & \normalsize{\jahrgang} \tabularnewline
	\textbf{\normalsize{Semester:}} & \normalsize{\semester} 
	\tabularnewline
	\textbf{\normalsize{Ausbildungsbetrieb:}} & \normalsize{\company}  \tabularnewline
    \textbf{\normalsize{Betreuender Prüfer:}} & \normalsize{\betreuerHS} \tabularnewline
    \textbf{\normalsize{Betreuer Betrieb:}} & \normalsize{\betreuerUnt} \tabularnewline
    \textbf{\normalsize{Unterschrift Betreuer: }} & \underline{\hspace{3cm}}
	%\textbf{\normalsize{Erstgutachter:}} & \normalsize{\betreuerUnt} \tabularnewline
	%\textbf{\normalsize{Zweitgutachter:}} & \normalsize{\betreuerHS}
	%\tabularnewline
	\end{tabular}
	}

\titlehead{\begin{center}
    \includegraphics{assets/images/HWR_Logo_farbig.jpg}
	\hfill
	\includegraphics[scale=0.05]{assets/images/DB_Logo.png}
	\vspace{2.5cm}
    \end{center}
}

\maketitle

\onehalfspacing % anderthalbfacher Zeilenabstand

%Abstract (Optional für PTB etc.)
\chapter*{Abstract} %dank dem Sternchen wird nicht Numeriert 

Das hier ist nur ein Beispieltext, der ein möglichen abstract darstellen soll. 
Falls du das nicht brauchst kannst du es ja rausnehmen. \par

Nur so als Beispiel noch ein paragraph.....
\addcontentsline{toc}{chapter}{Abstract}

%Inhaltsverzeichnis 
\tableofcontents{}
\addcontentsline{toc}{chapter}{Inhaltsverzeichnis}

\clearpage

\deftranslation[to=German]{Glossary}{Glossar}
\deftranslation[to=German]{Acronyms}{Akronyme}
\printnoidxglossaries

\clearpage

%%%%Abbildungsverzeichnis(If needed)

\listoffigures
\newpage

%%%%Tabellenverzeichnis(If needed)

\listoftables
\newpage

%Arabische Nummrierung 
\pagenumbering{arabic}

% Einleitung
\chapter{Einleitung}

\section{Begriffe}

Hier könnte deine Definition der wichtigsten Begriffe stehen. 
Wie zum Beispiel: \gls{AWS} oder \gls{BKU}
Andererseits haben wir ja noch Akronyme wie: 
\gls{DB} oder \gls{UX}
Nochmal ein versuch: \gls{DB} undso weiter

%Hauptteil
\chapter{Hauptteil}
Im Hauptteil ist es natürlich wichtig auch mit Fußnoten zu arbeiten. 
Wenn man zum beispiel behauptet, dass LaTex ein Softwarepaket ist, das die Benutzung des Textsatzsystems TeX mit Hilfe von Makros vereinfacht.\footnote{Vgl. Wikimedia foundation 2020}
Wusstest du das LaTex auch soviel bedeutet wir Lamport Tex ?\footnote{Vgl. ebd. 2020}

\section{Beispieltext}

Lorem ipsum dolor sit amet, consetetur sadipscing elitr, sed diam nonumy eirmod tempor invidunt ut labore et dolore magna aliquyam erat, sed diam voluptua. At vero eos et accusam et justo duo dolores et ea rebum. Stet clita kasd gubergren, no sea takimata sanctus est Lorem ipsum dolor sit amet. Lorem ipsum dolor sit amet, consetetur sadipscing elitr, sed diam nonumy eirmod tempor invidunt ut labore et dolore magna aliquyam erat, sed diam voluptua. At vero eos et accusam et justo duo dolores et ea rebum. Stet clita kasd gubergren, no sea takimata sanctus est Lorem ipsum dolor sit amet. Lorem ipsum dolor sit amet, consetetur sadipscing elitr, sed diam nonumy eirmod tempor invidunt ut labore et dolore magna aliquyam erat, sed diam voluptua. At vero eos et accusam et justo duo dolores et ea rebum. Stet clita kasd gubergren, no sea takimata sanctus est Lorem ipsum dolor sit amet.   

Duis autem vel eum iriure dolor in hendrerit in vulputate velit esse molestie consequat, vel illum dolore eu feugiat nulla facilisis at vero eros et accumsan et iusto odio dignissim qui blandit praesent luptatum zzril delenit augue duis dolore te feugait nulla facilisi. Lorem ipsum dolor sit amet, consectetuer adipiscing elit, sed diam nonummy nibh euismod tincidunt ut laoreet dolore magna aliquam erat volutpat.   

Ut wisi enim ad minim veniam, quis nostrud exerci tation ullamcorper suscipit lobortis nisl ut aliquip ex ea commodo consequat. Duis autem vel eum iriure dolor in hendrerit in vulputate velit esse molestie consequat, vel illum dolore eu feugiat nulla facilisis at vero eros et accumsan et iusto odio dignissim qui blandit praesent luptatum zzril delenit augue duis dolore te feugait nulla facilisi.   

Nam liber tempor cum soluta nobis eleifend option congue nihil imperdiet doming id quod mazim placerat facer

\section{Anderer wichtiger Standpunkt}

Lorem ipsum dolor sit amet, consetetur sadipscing elitr, sed diam nonumy eirmod tempor invidunt ut labore et dolore magna aliquyam erat, sed diam voluptua. At vero eos et accusam et justo duo dolores et ea rebum. Stet clita kasd gubergren, no sea takimata sanctus est Lorem ipsum dolor sit amet. Lorem ipsum dolor sit amet, consetetur sadipscing elitr, sed diam nonumy eirmod tempor invidunt ut labore et dolore magna aliquyam erat, sed diam voluptua. At vero eos et accusam et justo duo dolores et ea rebum. Stet clita kasd gubergren, no sea takimata sanctus est Lorem ipsum dolor sit amet. Lorem ipsum dolor sit amet, consetetur sadipscing elitr, sed diam nonumy eirmod tempor invidunt ut labore et dolore magna aliquyam erat, sed diam voluptua. At vero eos et accusam et justo duo dolores et ea rebum. Stet clita kasd gubergren, no sea takimata sanctus est Lorem ipsum dolor sit amet.   

Duis autem vel eum iriure dolor in hendrerit in vulputate velit esse molestie consequat, vel illum dolore eu feugiat nulla facilisis at vero eros et accumsan et iusto odio dignissim qui blandit praesent luptatum zzril delenit augue duis dolore te feugait nulla facilisi. Lorem ipsum dolor sit amet, consectetuer adipiscing elit, sed diam nonummy nibh euismod tincidunt ut laoreet dolore magna aliquam erat volutpat.   

Ut wisi enim ad minim veniam, quis nostrud exerci tation ullamcorper suscipit lobortis nisl ut aliquip ex ea commodo consequat. Duis autem vel eum iriure dolor in hendrerit in vulputate velit esse molestie consequat, vel illum dolore eu feugiat nulla facilisis at vero eros et accumsan et iusto odio dignissim qui blandit praesent luptatum zzril delenit augue duis dolore te feugait nulla facilisi.   

Nam liber tempor cum soluta nobis eleifend option congue nihil imperdiet doming id quod mazim placerat facer possim assum. Lorem ipsum dolor sit amet, consectetuer adipiscing elit, sed diam nonummy nibh euismod tincidunt ut laoreet dolore magna aliquam erat volutpat. Ut wisi enim ad minim veniam, quis nostrud exerci tation ullamcorper suscipit lobortis nisl ut aliquip ex ea commodo consequat.   

Duis autem vel eum iriure dolor in hendrerit in vulputate velit esse molestie consequat, vel illum dolore eu feugiat nulla facilisis.   

At vero eos et accusam et justo duo dolores et ea rebum. Stet clita kasd gubergren, no sea takimata sanctus est Lorem ipsum dolor sit amet. Lorem ipsum dolor sit amet, conset

\subsection{Teil davon}

Ich denke das sollte dann auch reichen.
Lorem ipsum dolor sit amet, consetetur sadipscing elitr, sed diam nonumy eirmod tempor invidunt ut labore et dolore magna aliquyam erat, sed diam voluptua. At vero eos et accusam et justo duo dolores et ea rebum. Stet clita kasd gubergren, no sea takimata sanctus est Lorem ipsum dolor sit amet. Lorem ipsum dolor sit amet, consetetur sadipscing elitr, sed diam nonumy eirmod tempor invidunt ut labore et dolore magna aliquyam erat, sed diam voluptua. At vero eos et accusam et justo duo dolores et ea rebum. Stet clita kasd gubergren, no sea takimata sanctus est Lorem ipsum dolor sit amet.

\section{Letzte wichtige Untersuchung}

Kommen wir nun zum Schluss...
\newpage

\end{document}
